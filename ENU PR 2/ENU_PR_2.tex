\newcommand{\institut}{Institut f\"ur Telekommunikationssysteme}
\newcommand{\fachgebiet}{Nachrichten\"ubertragung}
\newcommand{\veranstaltung}{Praktikum Nachrichten\"ubertragung}
\newcommand{\pdfautor}{Dirk Babendererde (321 836), Thomas Kapa (325 219)}
\newcommand{\autor}{Dirk Babendererde (321 836)\\ Thomas Kapa (325 219)}
\newcommand{\gruppe}{Gruppe:}
\newcommand{\betreuer}{Betreuer: Lieven Lange}


\newcommand{\pdftitle}{Nachrichten\"ubertragung\ Praktikum\ 02}
\newcommand{\prototitle}{Praktikum 02 \\ Statistische Nachrichtentheorie}



\input{../../packages/tu_header_8}


% \lstlistoflistings
\definecolor{darkgray}{rgb}{0.95,0.95,0.95}
\lstset{language=Scilab}
\lstset{inputencoding=utf8}
%\lstset{extendedchars=true} % Umlaute an der richtigen stelle und nicht am Anfang ausgeben
\lstset{backgroundcolor=\color{darkgray}}
\lstset{numbers=left, numberstyle=\tiny, stepnumber=1, numbersep=7pt, breaklines=true}
\lstset{keywordstyle=\color{red}\bfseries\emph}
\lstset{
breaklines,
numbers=left,
frame=single,
xleftmargin=-2cm,
xrightmargin=-1.5cm
}
% enables UTF-8 in source code: (dirty, dirty hack)
\lstset{literate=
    %Deutsch
    {ä}{{\"a}}1 {ö}{{\"o}}1 {ü}{{\"u}}1 {Ä}{{\"A}}1 {Ö}
    {{\"O}}1 {Ü}{{\"U}}1 {ß}{\ss}1
    %Türkisch
    {â}{{\^{a}}}1 {Â}{{\^{A}}}1 {ç}{{\c{c}}}1 {Ç}{{\c{C}}}1 {ğ}{{\u{g}}}1 {Ğ}{{\u{G}}}1 {ı}{{\i}}1 {İ}{{\.{I}}}1 {ö}{{\"o}}1 {Ö}{{\"O}}1 {ş}{{\c{s}}}1
    {Ş}{{\c{S}}}1 {ü}{{\"u}}1 {Ü}{{\"U}}1
    %Polish
    {ą}{{\k{a}}}1 {ć}{{\'c}}1 {ę}{{\k{e}}}1 {ł}{{\l{}}}1 {ń}{{\'n}}1 {ó}{{\'o}}1 {ś}{{\'s}}1 {ż}{{\.z}}1 {ź}{{\'z}}1 {Ą}{{\k{A}}}1 {Ć}{{\'C}}1
    {Ę}{{\k{E}}}1 {Ł}{{\L{}}}1 {Ń}{{\'N}}1 {Ó}{{\'O}}1 {Ś}{{\'S}}1 {Ż}{{\.Z}}1 {Ź}{{\'Z}}1
    %Spanish
    {á}{{\'a}}1 {é}{{\'e}}1 {í}{{\'i}}1 {ó}{{\'o}}1 {ú}{{\'u}}1 {ñ}{{\~n}}1
}

%     \lstinputlisting{./praktikum6.sce}



%---------------------------------------------------------------------
%---------------------------------------------------------------------
%---------------------------------------------------------------------

\section{Vorbereitungsaufgaben}

    \begin{quote}
    Berechnet die Varianz von gleichverteiltem weißem Rauschen N mit der
    Verteilungsdichtefunktion $p_{N}(n)=\frac{1}{2A}\sqcap_{2A}(n)$. Zunächst
    berechnet man den Mittelwert und die Leistung und kann daraus die Varianz
    bestimmen.
    \begin{equation*}
     \begin{split}
     \mu &= \int_{-\infty}^{+\infty} n \frac{1}{2A} \sqcap_{2A} (n) \mathrm dn\\
     &= \int_{-A}^{+A} n \frac{1}{2A} \mathrm dn\\
     &= \left[ \frac{1}{2A} \frac{1}{2} n^2 \right]_{-A}^{+A}\\
     &= \frac{1}{4A} (A^2-(-A)^2)\\
     &= 0
     \end{split}
    \end{equation*}
    
    \begin{equation*}
     \begin{split}
     P &= \int_{-\infty}^{+\infty} n^2 \frac{1}{2A} \sqcap_{2A} (n) \mathrm dn\\
     &= \int_{-A}^{+A} n^2 \frac{1}{2A} \mathrm dn\\
     &= \left[ \frac{1}{2A} \frac{1}{3} n^3 \right]_{-A}^{+A}\\
     &= \frac{1}{6A} (A^3-(-A)^3)\\
            &= \frac{2A^3}{6A} = \frac{1}{3} A^2\\
     \end{split}
    \end{equation*}
    
    \begin{equation*}
     \begin{split}
     \sigma^2 &= P - \mu^2\\
     &= \frac{1}{3} A^2 - 0 = \frac{1}{3} A^2
     \end{split}
    \end{equation*}
\end{quote}

\end{document}
